%\documentstyle[epsf,twocolumn]{jarticle}       %LaTeX2e仕様
\documentclass[twocolumn]{jarticle}     %pLaTeX2e仕様(platex.exeの場合)
% \documentclass[onecolumn]{ujarticle}   %pLaTeX2e仕様(uplatex.exeの場合)
%%%%%%%%%%%%%%%%%%%%%%%%%%%%%%%%%%%%%%%%%%%%%%%%%%%%%%%%%%%%%%
%%
%%  基本バージョン
%%
%%%%%%%%%%%%%%%%%%%%%%%%%%%%%%%%%%%%%%%%%%%%%%%%%%%%%%%%%%%%%%%%
\setlength{\topmargin}{-45pt}
%\setlength{\oddsidemargin}{0cm}
\setlength{\oddsidemargin}{-7.5mm}
%\setlength{\evensidemargin}{0cm}
\setlength{\textheight}{24.1cm}
%setlength{\textheight}{25cm}
\setlength{\textwidth}{17.4cm}
%\setlength{\textwidth}{172mm}
\setlength{\columnsep}{11mm}

%\kanjiskip=.07zw plus.5pt minus.5pt


% 【節が変わるごとに (1.1)(1.2) … (2.1)(2.2) と数式番号をつけるとき】
%\makeatletter
%\renewcommand{\theequation}{%
%\thesection.\arabic{equation}} %\@addtoreset{equation}{section}
%\makeatother

%\renewcommand{\arraystretch}{0.95} 行間の設定
%%%%%%%%%%%%%%%%%%%%%%%%%%%%%%%%%%%%%%%%%%%%%%%%%%%%%%%%
%\usepackage{graphicx}   %pLaTeX2e仕様(\documentstyle ->\documentclass)
\usepackage[dvipdfmx]{graphicx}
\usepackage{subcaption}
\usepackage{multirow}
\usepackage{amsmath}
\usepackage{url}
\usepackage{ulem}
\usepackage{algorithm}
\usepackage{algorithmic}
\usepackage{listings} %,jlisting} %日本語のコメントアウトをする場合jlistingが必要
%ここからソースコードの表示に関する設定
\lstset{
  basicstyle={\ttfamily},
  identifierstyle={\small},
  commentstyle={\smallitshape},
  keywordstyle={\small\bfseries},
  ndkeywordstyle={\small},
  stringstyle={\small\ttfamily},
  frame={tb},
  breaklines=true,
  columns=[l]{fullflexible},
  numbers=left,
  xrightmargin=0zw,
  xleftmargin=3zw,
  numberstyle={\scriptsize},
  stepnumber=1,
  numbersep=1zw,
  lineskip=-0.5ex
}
%%%%%%%%%%%%%%%%%%%%%%%%%%%%%%%%%%%%%%%%%%%%%%%%%%%%%%%%
\begin{document}

	%bibtex用の設定
	%\bibliographystyle{ujarticle}

	\twocolumn[
		\noindent
		\hspace{1em}
		2020 年 7 月 3 日
		ゼミ資料
		\hfill
		B4 杉山 竜弥
		\vspace{2mm}

		\hrule
		\begin{center}
			{\Large \bf 進捗報告}
		\end{center}
		\hrule
		\vspace{9mm}
	]

	% ‚ここから 文章 Start!
\section{今週やったこと}
\begin{itemize}
	\item {論文読んだ}
	\item {CNNの縮小}
\end{itemize}

\section{論文}
貢献度分配を導入した方策勾配によるNeural Architecture Searchの高速化

\subsection{はじめに}
アーキテクチャ最適化は, 学習に時間がかかる
\paragraph{One-shotアーキテクチャ}
アーキテクチャとパラメータを同時学習.
アーキテクチャを連続的に表現して、微分可能にする.

\subsection{Neural Architecture Search}
\subsubsection{有効非巡回グラフとしてのDNN}
\begin{description}
  \item[ノード] 特徴$x$
  \item[エッジ] オペレーション$O$
  \item[$a_i$] エッジ$i$のアーキテクチャ. $|a_i|=1$のときone-hotベクトル.
\end{description}

\subsubsection{One-shot Architecture Search}
\begin{description}
  \item[$R(a; w_a)$] 評価関数. ここでは負の交差エントロピー誤差.
  \item[$w_a$] パラメータ. $\hat{w}$に保存した学習済みパラメータから取得することで学習コストを削減.
\end{description}

\subsection{関連研究}
\subsubsection{ENAS}
REINFORCE
\begin{description}
  \item[目的関数] $E_{a	\sim p_\theta{(a)}} [R(a)]$%, $argmax \theta$
  \item[アーキテクチャ] $a$
\end{description}

\subsubsection{DARTS}
連続値アーキテクチャ
\begin{description}
  \item[評価関数] $E[R(\mu(\theta))]$
  \item[アーキテクチャ] $\theta \in [0, 1]^{|I||J|}$
\end{description}
学習後確率分布$\theta$からone-hot化する\\
\[a_{ij} = \lim_{\lambda \rightarrow 0} [\mu_{ij} (\theta / \lambda)]\]
\[j = {\arg\max}_j \mu_{ij}(\theta)\]

\subsubsection{SNAS}
連続値アーキテクチャ
\begin{description}
  \item[評価関数] $E_G [R(m(\mu_{ij}(\theta), G))]$
  \item[アーキテクチャ] DARTS同様
  \item[$G$] 標準Gumbel乱数
  \item[$m_{ij}(\theta, G)$] $\mu_{ij}(\theta)$に乱数$G$と温度パラメータ$\lambda$を導入
\end{description}

\subsubsection{ProxylessNAS}
\begin{description}
  \item[評価関数] $E_N [R(z(\mu(\theta), N))]$
  \item[アーキテクチャ] DARTS同様
  \item[$z_i(\mu(\theta), N)$] 乱数$N$からtwo-hotベクトルを生成
\end{description}

% \subsubsection{計算コストの比較}


\subsection{提案手法}
\begin{itemize}
  \item DARTS, SNAS\\オペレーション候補ごとに計算. $O(|J|)$.
  \item 提案手法\\貢献度分配で効率化.$|J|$に依存しない
\end{itemize}

\subsubsection{貢献度}
各エッジごとに, 全体に与える評価$R(a)$の善し悪しがある.\\
各演算子を一度に更新できる\\
貢献度$A$:全体評価$R(a)$に与える影響. 各エッジの評価
\[A(a_i, S_i) = R(a_i, S_i) - R(\vec{0}, S_i)\]
% \[S_i = {a_g|g \in I, g \neq i}\]
$S_i$:$a$から要素$a_i$を除いた集合

\subsubsection{効率的な貢献度の計算}
貢献度$A$を近似して$R$の評価回数を減らす
(13)
(14)

\subsection{実験}
% \subsubsection{Toy Problem}
% \begin{description}
%   \item[手法] REINFORCE, DARTS, SNAS, ProxylessNAS, 提案手法
%   \item[教師モデル] 2層全結合ネット(input 500, hidden 500, output 2)、活性化関数はランダムに決定
%   \item[生徒モデル 教師モデルと同じ重み, 活性化関数を探索
%   \item[活性化関数] x, x^2, x^3, ReLU(x), sigmoid(x)
%   \item[損失関数] 教師モデルと生徒モデルの出力の交差エントロピー誤差
% \end{description}
%
% 5.1.1. 結果

\subsubsection{アーキテクチャ探索}
\begin{itemize}
  \item セル\\
        (エッジ14, ノード7)$\times 8$個
  \item ノード\\
        c-1, c-2番目のセルからの入力, c番目のセルの出力, 中間ノード$\times 4$
  \item エッジ候補\\
        3x3 / 5x5 sep conv, 3x3 / 5x5 dilated sep conv, 3x3 max pooling, 3x3 average pooling, 恒等写像, 零写像の8つ%(チャンネル数 16)
  \item 学習\\
        $\sim$ 50 epoch(重み学習), 50 epoch $\sim$ 150 epoch($\theta$も更新)
\end{itemize}

\paragraph{結果}
得られた$\theta$をone-hot化し、重みを再学習
\begin{itemize}
  \item DARTS, SNASに迫る精度
  \item DARTS, SNASの3割程度の学習時間
\end{itemize}

\section{考察}
% 精度が出ない,とかだけではなく自分なりの考察を示す

\begin{itemize}
  \item アーキテクチャをカテゴリカルから連続的な確率分布に
  \item ネットワークの重みを再利用して学習を削減
  \item 見つけたアーキテクチャの重みを再学習する(acc 6割 $\rightarrow$ 9割)
\end{itemize}

\paragraph{重みの再利用}
\begin{enumerate}
  \item 冗長にネットワーク構造を決めておく. (あるノードはそれ以前のノード全てに接続可能とする)
  \item 重みを学習して、各エッジ、各演算子ごとに重みを保存
  \item アーキテクチャ(接続するか?+演算子)を探索
\end{enumerate}

得られたセルでは,
直前のセルからの入力を恒等写像でそのまま出力していた.

畳み込み層は全てseparable convolutionを利用している. パラメータ数を減らしたかった?

\paragraph{アーキテクチャの表現法}
\begin{itemize}
  \item RNNでパラメータを生成(カテゴリカル)
  \item GAで個体表現(カテゴリカル)
  \item 演算子の確率分布集合として表現
\end{itemize}

\section{RNNの動作実験}
目的:RNNでパラメータを生成

\subsection{}
input:系列データ[1, 0, 1, 0]\\
output:系列データ[1, 1, 0, 0]\\
lossはほぼ0になり学習できた

\subsection{}
直前の出力を入力することで再帰的にパラメータを出力\\
input:初期乱数z\\
output:系列データ[1, 1, 0, 0]\\
出力が常に0となり, 学習できなかった.

\section{今後の予定}
% なんとなくなんかの勉強をするとかではなく具体的に
\begin{itemize}
	\item {RNNの再帰的な入力の学習実験}
\end{itemize}


\section{ソースコード}
% 埋め込みでもGitでもいいので参照できるように
% \url{https://github.com/tatsuya-sugiyama/WeeklyReport/blob/2020_0626/report/2020_06_26/5%20to%2010.ipynb}
% % \begin{lstlisting}[caption=concat,label=model_cnn]
% % \end{lstlisting}


% 参考文献リスト
\bibliographystyle{unsrt}
\bibliography{ref}
\end{document}
