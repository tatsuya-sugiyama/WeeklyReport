%\documentstyle[epsf,twocolumn]{jarticle}       %LaTeX2e仕様
\documentclass[twocolumn]{jarticle}     %pLaTeX2e仕様(platex.exeの場合)
% \documentclass[onecolumn]{ujarticle}   %pLaTeX2e仕様(uplatex.exeの場合)
%%%%%%%%%%%%%%%%%%%%%%%%%%%%%%%%%%%%%%%%%%%%%%%%%%%%%%%%%%%%%%
%%
%%  基本バージョン
%%
%%%%%%%%%%%%%%%%%%%%%%%%%%%%%%%%%%%%%%%%%%%%%%%%%%%%%%%%%%%%%%%%
\setlength{\topmargin}{-45pt}
%\setlength{\oddsidemargin}{0cm}
\setlength{\oddsidemargin}{-7.5mm}
%\setlength{\evensidemargin}{0cm}
\setlength{\textheight}{24.1cm}
%setlength{\textheight}{25cm}
\setlength{\textwidth}{17.4cm}
%\setlength{\textwidth}{172mm}
\setlength{\columnsep}{11mm}

%\kanjiskip=.07zw plus.5pt minus.5pt


% 【節が変わるごとに (1.1)(1.2) … (2.1)(2.2) と数式番号をつけるとき】
%\makeatletter
%\renewcommand{\theequation}{%
%\thesection.\arabic{equation}} %\@addtoreset{equation}{section}
%\makeatother

%\renewcommand{\arraystretch}{0.95} 行間の設定
%%%%%%%%%%%%%%%%%%%%%%%%%%%%%%%%%%%%%%%%%%%%%%%%%%%%%%%%
%\usepackage{graphicx}   %pLaTeX2e仕様(\documentstyle ->\documentclass)
\usepackage[dvipdfmx]{graphicx}
\usepackage{subcaption}
\usepackage{multirow}
\usepackage{amsmath}
\usepackage{url}
\usepackage{ulem}
\usepackage{algorithm}
\usepackage{algorithmic}
\usepackage{listings} %,jlisting} %日本語のコメントアウトをする場合jlistingが必要
%ここからソースコードの表示に関する設定
\lstset{
  basicstyle={\ttfamily},
  identifierstyle={\small},
  commentstyle={\smallitshape},
  keywordstyle={\small\bfseries},
  ndkeywordstyle={\small},
  stringstyle={\small\ttfamily},
  frame={tb},
  breaklines=true,
  columns=[l]{fullflexible},
  numbers=left,
  xrightmargin=0zw,
  xleftmargin=3zw,
  numberstyle={\scriptsize},
  stepnumber=1,
  numbersep=1zw,
  lineskip=-0.5ex
}
%%%%%%%%%%%%%%%%%%%%%%%%%%%%%%%%%%%%%%%%%%%%%%%%%%%%%%%%
\begin{document}

	%bibtex用の設定
	%\bibliographystyle{ujarticle}

	\twocolumn[
		\noindent
		\hspace{1em}
		2020 年 5 月 29 日
		ゼミ資料
		\hfill
		B4 杉山 竜弥
		\vspace{2mm}

		\hrule
		\begin{center}
			{\Large \bf 進捗報告}
		\end{center}
		\hrule
		\vspace{9mm}
	]

	% ‚ここから 文章 Start!
\section{今週やったこと}
\begin{itemize}
	\item {論文読んでまとめた}
\end{itemize}



\section{CMA-ES に基づく適応度景観推定型進化型
計算の提案}
% \subsection{概要}
%
% \begin{enumerate}
%   \item 単純な生成評価法
%   \item 反復的な生成評価法
%   \item 高レベルな生成評価法
% \end{enumerate}

\subsection{導入}
進化型計算 Evolutionary Computation:EC
利点:汎用性高い
欠点:適応度評価に時間がかかる
→目的:評価回数の削減

\subsection{遺伝的アルゴリズム}
解を個体で表現

個体表現
\begin{itemize}
  \item 遺伝子の記号列
  \item 位置:遺伝子座
  \item 候補:対立遺伝子
  \item 長さ:遺伝子長
\end{itemize}

選択
・優れた解近傍の探索の重点化
・適応度で次世代の個体を決定
探索と多様性(初期収束問題)のトレードオフ
ルーレット戦略
適応度に比例して選択
トーナメント戦略
トーナメントサイズ分ランダム抽出し、最良個体を選択
エリート保存戦略
最も適応度の高い個体を変更せずに残す

交叉
親の染色体から子を生成
一点交叉
多点交叉
一様交叉

突然変異
ランダムな遺伝子座を対立遺伝子に置換
多様性、解の破壊

単純遺伝的アルゴリズムSGA
1.
2.ランダム生成
3.エリート選択
4.選択(ルーレット戦略)、交叉(2つの親から2つの子)、突然変異、エリートと入れ替え
5.終了条件

\subsection{進化戦略}
個体が実数ベクトル
交叉ではなく、正規分布による摂動
利点:遺伝子型から表現型への変換が不要

μ:親
λ:子

(μ, λ)-ES
次世代は子個体群からμ個選択
世代交代する多点探索

(μ+λ)-ES
次世代は親+子からμ個選択
エリート保存する多点探索
(1+1)-ES
山登り法
(1+λ)-ES
1点のみ保持、周辺探索
(μ+1)-ES
新しい解は1つだけ、連続世代


CMA-ES
多変量正規分布の母数(平均ベクトル・共分散行列)を効果的に更新
利点:変数の相互作用に強い

更新
平均ベクトル:Recombination
突然変異(分散?)の大きさ:Step Size  Adaptation:SSA
共分散行列:Cocariance Matrixx Adaptation:CMA

初期化
サンプリング
子個体生成
N(m, σ^2C)
平均ベクトルを加重平均に更新
進化パスで大域ステップサイズを更新
分散の大きさは適応的に変化、共分散と分離
進化パスで共分散行列を更新(進化パス:平均の移動量、共分散行列:上位μ個体を利用)
終了条件

\subsection{適応度景観推定型進化型計算(Fitness Landscape Learning Evolutionary Computation:FLLEC)}

個体の適応度計算(Fitness function)は時間がかかる
適応度を推定するサロゲート(代理)モデルで短縮

Rank Space Estimationモデル:RSE
比較指標となる適用度ではなく、1対比較を予測するモデル。学習が容易。
ここではSVMを使用

実装の考慮点
・選択アルゴリズム
トーナメント:OK
適応度上位を選択:難しい
・モデルの利用法
世代基準適用法
一定世代経過で、全ての個体の適応度を評価し、モデルを更新
個体基準適用法
常に一部を選んで適応度を評価し、一定世代ごとにモデルを更新
・モデル再学習の頻度


Air GA
RSEを利用。連続値への拡張
1.初期化
2.初期ランダム個体生成 適応度計算
3.交叉、突然変異
4.適応度評価して保存、エリート保存
5.保存した遺伝子・適応度で、一定世代ごとにSVMを再学習
6.エリート個体+SVMでトーナメント選択した個体
7.終了条件


FLLECの連続値拡張の考慮点
・アルゴリズム(RSEモデル)
・解収束時のSVM識別率低下

\subsection{提案手法}
CMA-ES+RSEモデル
改善:分布中心の更新時の評価回数をRSEで削減・(SVMで判定=>誤識別、SVMで大まかに判定=>全体の評価数削減)

データのスケーリング
分布の偏りを一様分布に変換して効率的に探索
・シグモイド関数
・共分散行列Cの利用(逆行列)


\section{考察}
% 精度が出ない,とかだけではなく自分なりの考察を示す


\section{今後の予定}
% なんとなくなんかの勉強をするとかではなく具体的に
\begin{itemize}
	\item {}
\end{itemize}

% 参考文献リスト
\bibliographystyle{unsrt}
\bibliography{ref}
\end{document}
