%\documentstyle[epsf,twocolumn]{jarticle}       %LaTeX2e仕様
\documentclass[twocolumn]{jarticle}     %pLaTeX2e仕様(platex.exeの場合)
% \documentclass[onecolumn]{ujarticle}   %pLaTeX2e仕様(uplatex.exeの場合)
%%%%%%%%%%%%%%%%%%%%%%%%%%%%%%%%%%%%%%%%%%%%%%%%%%%%%%%%%%%%%%
%%
%%  基本バージョン
%%
%%%%%%%%%%%%%%%%%%%%%%%%%%%%%%%%%%%%%%%%%%%%%%%%%%%%%%%%%%%%%%%%
\setlength{\topmargin}{-45pt}
%\setlength{\oddsidemargin}{0cm}
\setlength{\oddsidemargin}{-7.5mm}
%\setlength{\evensidemargin}{0cm}
\setlength{\textheight}{24.1cm}
%setlength{\textheight}{25cm}
\setlength{\textwidth}{17.4cm}
%\setlength{\textwidth}{172mm}
\setlength{\columnsep}{11mm}

%\kanjiskip=.07zw plus.5pt minus.5pt


% 【節が変わるごとに (1.1)(1.2) … (2.1)(2.2) と数式番号をつけるとき】
%\makeatletter
%\renewcommand{\theequation}{%
%\thesection.\arabic{equation}} %\@addtoreset{equation}{section}
%\makeatother

%\renewcommand{\arraystretch}{0.95} 行間の設定
%%%%%%%%%%%%%%%%%%%%%%%%%%%%%%%%%%%%%%%%%%%%%%%%%%%%%%%%
%\usepackage{graphicx}   %pLaTeX2e仕様(\documentstyle ->\documentclass)
\usepackage[dvipdfmx]{graphicx}
\usepackage{subcaption}
\usepackage{multirow}
\usepackage{amsmath}
\usepackage{url}
\usepackage{ulem}
\usepackage{algorithm}
\usepackage{algorithmic}
\usepackage{listings} %,jlisting} %日本語のコメントアウトをする場合jlistingが必要
%ここからソースコードの表示に関する設定
\lstset{
  basicstyle={\ttfamily},
  identifierstyle={\small},
  commentstyle={\smallitshape},
  keywordstyle={\small\bfseries},
  ndkeywordstyle={\small},
  stringstyle={\small\ttfamily},
  frame={tb},
  breaklines=true,
  columns=[l]{fullflexible},
  numbers=left,
  xrightmargin=0zw,
  xleftmargin=3zw,
  numberstyle={\scriptsize},
  stepnumber=1,
  numbersep=1zw,
  lineskip=-0.5ex
}
%%%%%%%%%%%%%%%%%%%%%%%%%%%%%%%%%%%%%%%%%%%%%%%%%%%%%%%%
\begin{document}

	%bibtex用の設定
	%\bibliographystyle{ujarticle}

	\twocolumn[
		\noindent
		\hspace{1em}
		2020 年 6 月 5 日
		ゼミ資料
		\hfill
		B4 杉山 竜弥
		\vspace{2mm}

		\hrule
		\begin{center}
			{\Large \bf 進捗報告}
		\end{center}
		\hrule
		\vspace{9mm}
	]

	% ‚ここから 文章 Start!
\section{今週やったこと}
\begin{itemize}
	\item {論文読んでまとめた}
	\item {5クラス識別を組み合わせたモデルの構築}
\end{itemize}



\section{CMA-ES に基づく適応度景観推定型進化型
計算の提案}

\subsection{導入}
進化型計算 Evolutionary Computation:EC\\
利点:汎用性高い\\
欠点:適応度評価に時間がかかる\\
→目的:評価回数の削減

\subsection{遺伝的アルゴリズム}
解を個体で表現

\subparagraph{個体表現}
\begin{itemize}
  \item 遺伝子の記号列
  \item 位置:遺伝子座
  \item 候補:対立遺伝子
  \item 長さ:遺伝子長
\end{itemize}

\subparagraph{選択}
\begin{itemize}
  \item 優れた解近傍の探索の重点化
  \item 適応度で次世代の個体を決定
\end{itemize}
探索と多様性(初期収束問題)のトレードオフ

\begin{itemize}
  \item ルーレット戦略\\
  適応度に比例して選択
  \item トーナメント戦略\\
  トーナメントサイズ分ランダム抽出し、最良個体を選択
  \item エリート保存戦略\\
  最も適応度の高い個体を変更せずに残す
\end{itemize}

\subparagraph{交叉}
親の染色体から子を生成
\begin{itemize}
  \item 一点交叉
  \item 多点交叉
  \item 一様交叉
\end{itemize}

\subparagraph{突然変異}
ランダムな遺伝子座を対立遺伝子に置換\\
多様性と解の保存のトレードオフ

\subsubsection{単純遺伝的アルゴリズムSGA}
\begin{enumerate}
  \item  初期化
  \item ランダム生成
  \item エリート個体選択
  \item 選択(ルーレット戦略)、交叉(2つの親から2つの子)、突然変異、エリート個体と入れ替え
  \item 終了条件
\end{enumerate}


\subsection{進化戦略}
個体が実数ベクトル\\
交叉ではなく、正規分布による摂動\\
利点:遺伝子型から表現型への変換が不要

μが親、λが子\\

\paragraph{(μ, λ)-ES}
次世代は子個体群からμ個選択
世代交代する多点探索

\paragraph{(μ+λ)-ES}
次世代は親+子からμ個選択\\
エリート保存する多点探索
\begin{itemize}
  \item (1+1)-ES\\
  山登り法
  \item (1+λ)-ES\\
  1点のみ保持、周辺探索
  \item (μ+1)-ES\\
  新しい解は1つだけ、連続世代
\end{itemize}

\subsubsection{CMA-ES}
多変量正規分布の母数(平均ベクトル・共分散行列)を効果的に更新\\
利点:変数の相互作用に強い

% \paragraph{更新}
% \begin{itemize}
%   \item 平均ベクトル:Recombination
%   \item 突然変異(分散?)の大きさ:Step Size  Adaptation:SSA
%   \item 共分散行列:Cocariance Matrixx Adaptation:CMA
% \end{itemize}



\begin{enumerate}
  \item 初期化
  \item サンプリング
  \item 子個体生成\\
    $N(m, σ^2C)$
  \item 平均ベクトルを加重平均に更新
  \item 進化パスで大域ステップサイズを更新\\
    分散の大きさは適応的に変化、共分散と分離
  \item 進化パスで共分散行列を更新\\
      (進化パス:平均の移動量、共分散行列:上位μ個体を利用)
  \item 終了条件
\end{enumerate}


\subsection{適応度景観推定型進化型計算(Fitness Landscape Learning Evolutionary Computation:FLLEC)}
個体の適応度計算(Fitness function)は時間がかかる\\
適応度を推定するサロゲート(代理)モデルで短縮

\begin{itemize}
  \item 世代基準適用法\\
    一定世代経過で、全ての個体の適応度を評価し、モデルを更新
  \item 個体基準適用法\\
    常に一部を選んで適応度を評価し、一定世代ごとにモデルを更新
\end{itemize}

% \subsubsection{実装の考慮点}
% \paragraph{選択アルゴリズム}
% \begin{itemize}
%   \item トーナメント:OK
%   \item 適応度上位を選択:難しい
% \end{itemize}
%
% \paragraph{モデルの利用法}
%
% \paragraph{モデル再学習の頻度}

\subsubsection{Rank Space Estimationモデル:RSE}
比較指標となる適用度ではなく、1対比較を予測するモデル. 学習が容易.
ここではSVMを使用

\subsubsection{Air GA}
RSEを利用. 連続値への拡張
\begin{enumerate}
  \item 初期化
  \item 初期ランダム個体生成 適応度計算
  \item 交叉、突然変異
  \item 適応度評価して保存、エリート個体保存
  \item 保存した遺伝子・適応度で、一定世代ごとにSVMを再学習
  \item エリート個体 + SVMでトーナメント選択した個体
  \item 終了条件
\end{enumerate}

% FLLECの連続値拡張の考慮点
% \begin{itemize}
%   \item アルゴリズム(RSEモデル)
%   \item 解収束時のSVM識別率低下
% \end{itemize}


\subsection{提案手法}
CMA-ES+RSEモデル\\
改善:分布中心の更新時の評価回数をRSEで削減・(SVMで判定→誤識別、SVMで大まかに判定→全体の評価数削減)

\paragraph{データのスケーリング}
分布の偏りを一様分布に変換して効率的に探索
\begin{itemize}
  \item シグモイド関数
  \item 共分散行列Cの利用(逆行列)
\end{itemize}

\section{モデルの構築}
5クラス識別を組み合わせたモデルの構築. 今週は完成しなかった.

\section{考察}
% 精度が出ない,とかだけではなく自分なりの考察を示す
モデルを簡潔に書けるライブラリを導入しようと格闘したがうまくいかなかった. 一旦あきらめて, モデルの完成を目指す.

\section{今後の予定}
% なんとなくなんかの勉強をするとかではなく具体的に
\begin{itemize}
	\item {モデルの完成と実験}
\end{itemize}

% 参考文献リスト
\bibliographystyle{unsrt}
\bibliography{ref}
\end{document}
