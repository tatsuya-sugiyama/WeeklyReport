\newpage
\changeindent{0cm}
\section{はじめに}
\label{sec:intro}
\changeindent{2cm}

\subsection{研究背景}

機械学習の分野では, 深層学習モデルの改良によって大きく精度が向上してきた.
しかしモデルの設計とその性能の関係はブラックボックスであり
手作業でによるチューニングには膨大な労力を要する.

ネットワークの探索を自動化する手法として提案された
Neural Architecture Search(NAS)はネットワークを機械学習によって探索する.
しかし何千ものGPUを必要とするため, NASに代わり小規模な資源で計算できる
Differentiable Architecture Search(DARTS) が大きな注目を集めている.
DARTSはネットワークの構造と演算子の候補を探索するが,
一方でDARTSにはネットワーク構造にいくつかの拘束条件がある.

本研究では演算子の種類ではなくネットワークの構造にのみ着目し,
DARTSの構造制限をなくしネットワークの柔軟な探索を目的とする.
ベースとなるネットワークを19層構造のVGG19とし,
ショートカット位置についてDARTSで探索を行う方法と, それにTDGAを導入する方法を提案する.

\subsection{本論文の構成}

以下に本論文の構成を示す.まず,\ref{sec:tech} 章では本研究で用いる要素技術について概説する.
\ref{sec:pred} 章で深層学習の構造の設定と探索手法を提案する.
そして\ref{sec:exp} 章において,数値実験により手法の性能を検証し, 本研究で提案する手法の考察をする.
\ref{sec:conclusion} 章で本研究の成果をまとめたうえで,今後の課題について述べる.


\begin{comment}
\end{comment}
