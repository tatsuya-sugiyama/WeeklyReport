\newpage
\changeindent{0cm}
\section{はじめに}
\label{sec:intro}
\changeindent{2cm}

機械学習の分野では, 深層学習モデルの改良によって大きく精度が向上してきた.
しかしモデルの設計とその性能の関係はブラックボックスであり
手作業でによるチューニングには膨大な労力を要する.

ネットワークの探索を自動化する手法として提案された
Neural Architecture Search(NAS)はネットワークを機械学習によって探索する.
しかし何千ものGPUを必要とするため, NASに代わり小規模な資源で計算できる
Differentiable Architecture Search(DARTS) が大きな注目を集めている.
DARTSはネットワークの構造と演算子の候補を探索するが,
一方でDARTSにはネットワーク構造にいくつかの拘束条件がある.

また, ネットワークの構造を組み合わせ最適化と考える場合,
導入する手法として今回は遺伝的アルゴリズム (Genetic Algorithm: GA) に着目する.
GAは生物の進化の仕組みを模倣した最適化手法であるが,
初期段階で個体群が同じ個体で占められる初期収束問題がある.
一方Thermodynamical Genetic Algorithm (TDGA)は,
GAの選択ルールに多様性を考慮した適応度を導入することで初期収束問題を防ぐことができる.


本研究では演算子の種類ではなくネットワークの構造にのみ着目し,
DARTSの構造制限をなくしネットワークの柔軟な探索を目的とする.
ベースとなるネットワークを19層構造のVGG19とし,
ショートカット位置についてDARTSで探索を行う方法と, それにTDGAを導入する方法を提案する.


以下に本論文の構成を示す.まず,\ref{sec:tech} 章では本研究で用いる要素技術について概説する.
\ref{sec:pred} 章で深層学習の構造の設定と探索手法を提案する.
そして\ref{sec:exp} 章において,数値実験により手法の性能を検証し, 本研究で提案する手法の考察をする.
\ref{sec:conclusion} 章で本研究の成果をまとめたうえで,今後の課題について述べる.


\begin{comment}
\end{comment}
