\newpage
\changeindent{0cm}
\section{まとめと今後の課題}
\label{sec:conclusion}
\changeindent{2cm}

本研究では,
ネットワーク探索の自動化手法 DARTS について, その改良を検討した.

実験1では,
VGG19に対してDARTSを適用し, ネットワーク構造全体の接続を考え,
DARTSによって VGG19 の性能を改善することを目的とした.

結果としてベースラインやランダムに接続した場合よりも,
DARTSの探索したネットワーク構造が高い性能となり,
またDARTSの出力結果からネットワークを構築する手法についても実験し,
$\bm{\alpha}$の順序のみではなく, 辺ごとに$\bm{\alpha}$を閾値で採用する辺を構築する手法が,
より少ない辺でネットワークの性能を高めている事が分かった.

DARTSのために設計されたネットワーク以外に対してもネットワーク構造が探索でき,
DARTSの欠点といえるアーキテクチャ構造の制限を緩和できることが示された.

また実験2では,
DARTS に TDGAを組み合わせる手法を実験した.
実験1では, 複数回試行した実験で部分的な共通点は見られたものの,
全体としてはほとんど異なる構造となっていた.
このような初期値依存性による学習結果の偏りを解消するため,
GAによる多点探索を採用した.
GAで複数個体を同時に学習することで$\bm{\alpha}$や$\bm{w}$の学習しやすさを平均化することを考えた.
DARTSにおける$\bm{\alpha}$を個体とし, 多様性を維持するためTDGAを導入した.
また計算時間の問題も発生するため, 個体群全体で$\bm{w}$を共有するOne-Shotモデルとし,
$\bm{w}$の分の計算時間を削減した.
さらに2つの手法を組み合わせる際に, DARTSは実数値だが, TDGAは整数値しか扱えないという問題も
あったため, $\bm{\alpha}$の標準偏差でエントロピーの代わりとして, 多様性の維持をした.

実験2の結果として, DARTSのみの性能を提案手法が上回ることができ, 提案手法の有効性が確認できた.
3つの提案手法の中では, $\bm{w}$ を固定した提案手法2, 3がうまく学習でき,
TDGAの適応度計算に$\bm{w}$を更新するDARTSが与える悪影響が存在することが分かった.

またTDGAはDARTSに比べ計算時間が大幅に短いため,
TDGAで広範囲を探索した後, DARTSによって正確に探索するなどの今回の提案手法の改良も考えられた.


今後の課題として,
本研究で用いた VGG 以外のネットワークに対してもこの提案手法を適用することで汎用性を確認し,
また他のデータセットや実問題では, 最適なアーキテクチャがどのように変化するか検討することが挙げられる.
