\documentclass[a4paper,twoside,twocolumn,10pt]{article}
\usepackage{abstract} % Style for abstracts in Dept. CSIS, OPU
%\usepackage{abstract4past} % Style for abstracts for the past curriculum

%%%%%%%%%% Designate packages you need %%%%%%%%%%
\usepackage{graphicx} % Enhanced support for graphics
\usepackage{url} % Verbatim with URL-sensitive line breaks

%%%%%%%%%% Parameters that should be customized %%%%%%%%%%
% Language (1 = Japanese, 2 = English)
\setlang{1}
% Bachelor or Master (1 = Bachelor, 2 = Master)
\setborm{1}
% Fiscal year
\setfy{2018}
% Group number
\setgnum{1}
% Presentation order
\setorder{4}
% Increase page number (optional)
%% \pplus{1}

% Title
\title{タイトル未定}
% Author
\author{青野 義樹}
%%%%%%%%%% Parameters that should be customized (end) %%%%%%%%%%

\begin{document}
\maketitle % Insert title
\small

\section{はじめに}
この文書は,大阪府立大学 工学域 情報工学課程(旧カリキュラムである
工学部 知能情報工学科を含む)の卒業研究論文ならびに
大学院工学研究科 電気・情報系専攻 知能情報工学分野の修士学位論文の
概要テンプレートの使い方を説明した文書である.
この文書自体が概要テンプレートで書かれている.



\section{作成要領}

\subsection{卒業研究論文}
\begin{enumerate}
\item ページ制限,用紙\\
  卒業論文概要は A4 版 原則 1 ページとし片面を用いる(2人で両面).
  特別な場合2ページまで可.
\item フォーマット\\
  指定されたLaTeXまたはWordのテンプレートを使用する.
\item グループ番号とページ番号\\
  ページの右上に「グループ番号−グループ内通しページ番号」を記入する.
\end{enumerate}

\subsection{修士学位論文}
\begin{enumerate}
\item ページ制限,用紙\\
  修士学位論文概要は A4 版 2 ページとし両面を用いる.
\item フォーマット\\
  指定されたLaTeXまたはWordのテンプレートを使用する.
\item グループ番号とページ番号\\
  ページの右上に「グループ番号−グループ内通しページ番号」を記入する.
\end{enumerate}

\section{LaTeXテンプレート}
\subsection{ファイル}
\begin{itemize}
\item abstract\_ja.pdf\\
  サンプルPDF(日本語版)
\item abstract\_ja.tex\\
  サンプルLaTeXファイル(日本語版)
\item index\_ja.bib\\
  サンプルBibTeXファイル(日本語版)
\item abstract.sty\\
  概要スタイルファイル
\item jabbrvunsrt.bst\\
  BibTeXスタイルファイル(日本語版)
\item CSIS.eps\\
  分野のロゴ(図のサンプルとして)
\item fancyhdr.sty\\
  ヘッダとフッダを操作するスタイルファイル
  (\url{https://www.ctan.org/pkg/fancyhdr}よりダウンロード)
\item titlesec.sty\\
  セクションタイトルを操作するためのスタイルファイル
  (\url{https://www.ctan.org/pkg/titlesec}よりダウンロード)
\end{itemize}

\subsection{設定}
abstract.texの上部には以下の設定項目があり,各自しかるべき値に変更する.
%
\begin{verbatim}
% Language (1 = Japanese, 2 = English)
\setlang{1}
% Bachelor or Master (1 = Bachelor, 2 = Master)
\setborm{2}
% Fiscal year
\setfy{2015}
% Group number
\setgnum{3}
% Presentation order
\setorder{2}
% Increase page number (optional)
%% \pplus{1}
\end{verbatim}
%
Presentation orderは発表順である.
これが指定されると,卒業研究論文の場合は1人当たり1ページ,
修士学位論文の場合は1人当たり2ページとしてページ番号が自動的に計算される.
何らかの事情によりページ番号がずれる場合は,
$\backslash$pplus\{1\}
のように指定してページ番号を増加(または減少)させることができる.

\subsection{図表}
表~\ref{tbl:kuku}は表の例,
図~\ref{fig:CSIS_logo}は図の例である.

\begin{table}[tb]
  \caption{表の例:九九}
  \label{tbl:kuku}
  \centering
  \begin{tabular}{|c||c|c|c|c|c|c|c|c|c|} \hline
    - &  1 &  2 &  3 &  4 &  5 &  6 &  7 &  8 &  9 \\ \hline \hline
    1 &  1 &  2 &  3 &  4 &  5 &  6 &  7 &  8 &  9 \\ \hline
    2 &  2 &  4 &  6 &  8 & 10 & 12 & 14 & 16 & 18 \\ \hline
    3 &  3 &  6 &  9 & 12 & 15 & 18 & 21 & 24 & 27 \\ \hline
    4 &  4 &  8 & 12 & 16 & 20 & 24 & 28 & 32 & 36 \\ \hline
    5 &  5 & 10 & 15 & 20 & 25 & 30 & 35 & 40 & 45 \\ \hline
    6 &  6 & 12 & 18 & 24 & 30 & 36 & 42 & 48 & 54 \\ \hline
    7 &  7 & 14 & 21 & 28 & 35 & 42 & 49 & 56 & 63 \\ \hline
    8 &  8 & 16 & 24 & 32 & 40 & 48 & 56 & 64 & 72 \\ \hline
    9 &  9 & 18 & 27 & 36 & 45 & 54 & 63 & 72 & 81 \\ \hline
  \end{tabular}
\end{table}

\begin{figure}[tb]
  \centering
  \includegraphics[width=.3\hsize]{CSIS.eps}
  \caption{図の例:ロゴ}
  \label{fig:CSIS_logo}
\end{figure}

\subsection{参考文献}
pBibTeXの使用を推奨する.
その場合,同梱されているjabbrvunsrt.bstを使うこと.
これは,jabbrv.bstのソート機能をオフにしたものである.
\cite{SakaiMe, Food, Neko}は使用例である.

\subsection{旧カリキュラムの学生の場合}
旧カリキュラムである工学部 知能情報工学科に所属する学生は,
abstract.styの代わりにabstract4past.styを使用する.

\section{MS Wordテンプレート}
MS Word用のテンプレートも用意している.
フォーマットはLaTeXテンプレートに準ずることとする.
以下に設定方法の概要を示す。

\subsection{ファイル}
\begin{itemize}
\item abstract\_ja\_word.pdf\\
  サンプルPDF(日本語版)
\item abstract\_ja\_word.docx\\
  サンプルWordファイル(日本語版)
\end{itemize}

\subsection{ヘッダの設定}
ヘッダ部分をダブルクリックするとヘッダが編集可能になるので、以下の項目を設定する。

\subsubsection{論文の種類}
\begin{itemize}
\item 工学域 情報工学課程の場合\\
  「情報工学課程卒業研究論文概要」とする。
\item 大学院工学研究科 知能情報工学分野の場合\\
  「知能情報工学分野修士学位論文概要」とする。
\item 旧カリキュラムである工学部 知能情報工学科の場合\\
  「知能情報工学科卒業研究論文概要」とする。
\end{itemize}
\subsubsection{年度およびグループ番号}
年度およびグループ番号は直接入力する。
\subsubsection{ページ番号}
ページ番号は「ページ番号の書式設定」で「開始番号」を変更する。

\subsection{スタイル}
Normal, Title, Author, Section, SubSection, SubSubSection, References,
Table, Verbatim, Enumerate, Itemizeのスタイルが定義されているので、
適宜使用する。


%%%%%%%%%%%%%%%%%%%%%%%%%%%%%%

\bibliographystyle{jabbrvunsrt}
\bibliography{index_ja}
\end{document}
