%\documentstyle[epsf,twocolumn]{jarticle}       %LaTeX2e仕様
\documentclass[twocolumn]{jarticle}     %pLaTeX2e仕様(platex.exeの場合)
% \documentclass[onecolumn]{ujarticle}   %pLaTeX2e仕様(uplatex.exeの場合)
%%%%%%%%%%%%%%%%%%%%%%%%%%%%%%%%%%%%%%%%%%%%%%%%%%%%%%%%%%%%%%
%%
%%  基本バージョン
%%
%%%%%%%%%%%%%%%%%%%%%%%%%%%%%%%%%%%%%%%%%%%%%%%%%%%%%%%%%%%%%%%%
\setlength{\topmargin}{-45pt}
%\setlength{\oddsidemargin}{0cm}
\setlength{\oddsidemargin}{-7.5mm}
%\setlength{\evensidemargin}{0cm}
\setlength{\textheight}{24.1cm}
%setlength{\textheight}{25cm}
\setlength{\textwidth}{17.4cm}
%\setlength{\textwidth}{172mm}
\setlength{\columnsep}{11mm}

%\kanjiskip=.07zw plus.5pt minus.5pt


% 【節が変わるごとに (1.1)(1.2) … (2.1)(2.2) と数式番号をつけるとき】
%\makeatletter
%\renewcommand{\theequation}{%
%\thesection.\arabic{equation}} %\@addtoreset{equation}{section}
%\makeatother

%\renewcommand{\arraystretch}{0.95} 行間の設定
%%%%%%%%%%%%%%%%%%%%%%%%%%%%%%%%%%%%%%%%%%%%%%%%%%%%%%%%
%\usepackage{graphicx}   %pLaTeX2e仕様(\documentstyle ->\documentclass)
\usepackage[dvipdfmx]{graphicx}
\usepackage{subcaption}
\usepackage{multirow}
\usepackage{amsmath}
\usepackage{url}
\usepackage{ulem}
\usepackage{algorithm}
\usepackage{algorithmic}
\usepackage{listings} %,jlisting} %日本語のコメントアウトをする場合jlistingが必要
%ここからソースコードの表示に関する設定
\lstset{
  basicstyle={\ttfamily},
  identifierstyle={\small},
  commentstyle={\smallitshape},
  keywordstyle={\small\bfseries},
  ndkeywordstyle={\small},
  stringstyle={\small\ttfamily},
  frame={tb},
  breaklines=true,
  columns=[l]{fullflexible},
  numbers=left,
  xrightmargin=0zw,
  xleftmargin=3zw,
  numberstyle={\scriptsize},
  stepnumber=1,
  numbersep=1zw,
  lineskip=-0.5ex
}
%%%%%%%%%%%%%%%%%%%%%%%%%%%%%%%%%%%%%%%%%%%%%%%%%%%%%%%%
\begin{document}

	%bibtex用の設定
	%\bibliographystyle{ujarticle}

	\twocolumn[
		\noindent
		\hspace{1em}
		2020 年 7 月 17 日
		ゼミ資料
		\hfill
		B4 杉山 竜弥
		\vspace{2mm}

		\hrule
		\begin{center}
			{\Large \bf 進捗報告}
		\end{center}
		\hrule
		\vspace{9mm}
	]

	% ‚ここから 文章 Start!
\section{今週やったこと}
\begin{itemize}
	\item {NASの実装}
\end{itemize}

\section{NAS}
DARTS系のNASの実験をする.
ネットワークにはまずエッジとセルが必要となる.
コード1, 2にPytorchで実装したクラスを示した.
逆伝播ができることも確認した.


\section{今後の予定}
% なんとなくなんかの勉強をするとかではなく具体的に
実装したクラスでネットワークを構築して実験する.
対象とする問題はCifar-10を考えている.


\section{ソースコード}
% 埋め込みでもGitでもいいので参照できるように

動作することを第一に実装した.
operatorsで演算子の候補を受け取る.
セルは3つのノードで1入力1出力と暫定的に定めた.

\begin{lstlisting}[caption=Edge,label=Edge]
class Edge(nn.Module):
  def __init__(self, operators):
    super(Edge, self).__init__()
    self.operators = operators

    rand = torch.randn(len(operators), requires_grad=True)
    self.theta = rand / torch.sum(rand)

  def forward(self, input: Tensor) -> Tensor:
    output = torch.zeros(input.shape, requires_grad=True)
    for (theta_i, operator) in zip(self.theta, self.operators):
      if operator == None:
        continue
      output = output +  theta_i * operator(input)

    return output
\end{lstlisting}

\begin{lstlisting}[caption=Cell,label=Cell]
class Cell(nn.Module):
  def __init__(self, operators):
    super(Cell, self).__init__()

    self.node_num = 3

    self.ref = [(0, 1), (0, 2), (1, 2)]
    self.edges = [Edge(operators) for _ in self.ref]

  def forward(self, input) -> Tensor:
    nodes = [torch.zeros(*list(input.shape), requires_grad=True) for _ in range(self.node_num)]
    nodes[0] = input

    for idx, (inref, outref) in enumerate(self.ref):
      nodes[outref] = nodes[outref] + self.edges[idx](nodes[inref])

    return nodes[-1]
\end{lstlisting}


% 参考文献リスト
\bibliographystyle{unsrt}
\bibliography{ref}
\end{document}
