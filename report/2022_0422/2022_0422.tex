
%\documentstyle[epsf,twocolumn]{jarticle}       %LaTeX2e仕様
% \documentclass[twocolumn]{jarticle}     %pLaTeX2e仕様(platex.exeの場合)
\documentclass[onecolumn]{ujarticle}   %pLaTeX2e仕様(uplatex.exeの場合)
%%%%%%%%%%%%%%%%%%%%%%%%%%%%%%%%%%%%%%%%%%%%%%%%%%%%%%%%%%%%%%
%%
%%  基本バージョン
%%
%%%%%%%%%%%%%%%%%%%%%%%%%%%%%%%%%%%%%%%%%%%%%%%%%%%%%%%%%%%%%%%%
\setlength{\topmargin}{-45pt}
%\setlength{\oddsidemargin}{0cm}
\setlength{\oddsidemargin}{-7.5mm}
%\setlength{\evensidemargin}{0cm}
\setlength{\textheight}{24.1cm}
%setlength{\textheight}{25cm}
\setlength{\textwidth}{17.4cm}
%\setlength{\textwidth}{172mm}
\setlength{\columnsep}{11mm}

%\kanjiskip=.07zw plus.5pt minus.5pt


% 【節が変わるごとに (1.1)(1.2) … (2.1)(2.2) と数式番号をつけるとき】
%\makeatletter
%\renewcommand{\theequation}{%
%\thesection.\arabic{equation}} %\@addtoreset{equation}{section}
%\makeatother

%\renewcommand{\arraystretch}{0.95} 行間の設定
%%%%%%%%%%%%%%%%%%%%%%%%%%%%%%%%%%%%%%%%%%%%%%%%%%%%%%%%
%\usepackage{graphicx}   %pLaTeX2e仕様(\documentstyle ->\documentclass)
\usepackage[dvipdfmx]{graphicx}
\usepackage{subcaption}
\usepackage{multirow}
\usepackage{amsmath}
\usepackage{url}
\usepackage{ulem}
\usepackage{algorithm}
\usepackage{algorithmic}
\usepackage{listings} %,jlisting} %日本語のコメントアウトをする場合jlistingが必要
%ここからソースコードの表示に関する設定
\lstset{
  basicstyle={\ttfamily},
  identifierstyle={\small},
  commentstyle={\smallitshape},
  keywordstyle={\small\bfseries},
  ndkeywordstyle={\small},
  stringstyle={\small\ttfamily},
  frame={tb},
  breaklines=true,
  columns=[l]{fullflexible},
  numbers=left,
  xrightmargin=0zw,
  xleftmargin=3zw,
  numberstyle={\scriptsize},
  stepnumber=1,
  numbersep=1zw,
  lineskip=-0.5ex
}
\newcommand{\argmax}{\mathop{\rm arg~max}\limits}
\newcommand{\argmin}{\mathop{\rm arg~min}\limits}

%%%%%%%%%%%%%%%%%%%%%%%%%%%%%%%%%%%%%%%%%%%%%%%%%%%%%%%%
\begin{document}

	%bibtex用の設定
	%\bibliographystyle{ujarticle}

	% \twocolumn[
		\noindent
		\hspace{1em}
		2022 年 4 月 22 日
		ゼミ資料
		\hfill
		杉山 竜弥
		\vspace{2mm}

		\hrule
		\begin{center}
			{\Large \bf 進捗報告}
		\end{center}
		\hrule
		\vspace{9mm}
	% ]


\section{今週やったこと}
\begin{itemize}
  \item sketch-rnn 周辺の調査と動作実験
\end{itemize}

\section{SVG に関連する問題}
SVG に関するものを調査した. すぐに思いついたものはすでに先行研究があったので,研究の方向性は未定.
汎用的な問題は別の手法でも試され尽くしている気がするので,sketch rnn 系でなにかしたい.

\begin{itemize}
  \item 創作系  お絵かき\\
   sketch rnn\cite{ha2017neural}
  \item ラスタ画像からベクタ画像への変換\cite{liu2017raster}\\
      ベクタ画像からラスタ画像への逆変換は簡単なので,データセットを集めやすくていいと思ったが,
      すでに先行研究が存在した.
  \item フォントスタイル変換
  % \cite{kim2018semantic}
  \\
  日本語は文字数が多いので自動化できたら嬉しい.\\
   Deep Glyph(アプリ)\\
   ラスタ形式でStackGAN
\end{itemize}

\section{Sketch-Rnn}
動かせそうなコードがあったので,Sketch-Rnnの動作確認をした.
Sketch-Rnn は古いコードで, 現在のプロジェクトも tensor-flow v1系 を使用していたため,
コードの改造をする際に苦戦する可能性がある.

\section{データセット}
\begin{itemize}
  \item Quick Draw Dataset\\
  Quick Draw!というゲームで収集されたデータセット.
  データ数は 5000 万とのことで,手描きイラスト系ならこのデータセットで十分だと思われる.\\
  実験で使用できるように, 内部の形式や使用方法を調査したい.
\end{itemize}

% \section{今後の予定}
% % なんとなくなんかの勉強をするとかではなく具体的に
% \begin{itemize}
%   \item
% \end{itemize}

% 参考文献リスト
\bibliographystyle{unsrt}
\bibliography{ref}
\end{document}
