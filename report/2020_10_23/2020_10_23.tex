%\documentstyle[epsf,twocolumn]{jarticle}       %LaTeX2e仕様
\documentclass[twocolumn]{jarticle}     %pLaTeX2e仕様(platex.exeの場合)
% \documentclass[onecolumn]{ujarticle}   %pLaTeX2e仕様(uplatex.exeの場合)
%%%%%%%%%%%%%%%%%%%%%%%%%%%%%%%%%%%%%%%%%%%%%%%%%%%%%%%%%%%%%%
%%
%%  基本バージョン
%%
%%%%%%%%%%%%%%%%%%%%%%%%%%%%%%%%%%%%%%%%%%%%%%%%%%%%%%%%%%%%%%%%
\setlength{\topmargin}{-45pt}
%\setlength{\oddsidemargin}{0cm}
\setlength{\oddsidemargin}{-7.5mm}
%\setlength{\evensidemargin}{0cm}
\setlength{\textheight}{24.1cm}
%setlength{\textheight}{25cm}
\setlength{\textwidth}{17.4cm}
%\setlength{\textwidth}{172mm}
\setlength{\columnsep}{11mm}

%\kanjiskip=.07zw plus.5pt minus.5pt


% 【節が変わるごとに (1.1)(1.2) … (2.1)(2.2) と数式番号をつけるとき】
%\makeatletter
%\renewcommand{\theequation}{%
%\thesection.\arabic{equation}} %\@addtoreset{equation}{section}
%\makeatother

%\renewcommand{\arraystretch}{0.95} 行間の設定
%%%%%%%%%%%%%%%%%%%%%%%%%%%%%%%%%%%%%%%%%%%%%%%%%%%%%%%%
%\usepackage{graphicx}   %pLaTeX2e仕様(\documentstyle ->\documentclass)
\usepackage[dvipdfmx]{graphicx}
\usepackage{subcaption}
\usepackage{multirow}
\usepackage{amsmath}
\usepackage{url}
\usepackage{ulem}
\usepackage{algorithm}
\usepackage{algorithmic}
\usepackage{listings} %,jlisting} %日本語のコメントアウトをする場合jlistingが必要
%ここからソースコードの表示に関する設定
\lstset{
  basicstyle={\ttfamily},
  identifierstyle={\small},
  commentstyle={\smallitshape},
  keywordstyle={\small\bfseries},
  ndkeywordstyle={\small},
  stringstyle={\small\ttfamily},
  frame={tb},
  breaklines=true,
  columns=[l]{fullflexible},
  numbers=left,
  xrightmargin=0zw,
  xleftmargin=3zw,
  numberstyle={\scriptsize},
  stepnumber=1,
  numbersep=1zw,
  lineskip=-0.5ex
}
%%%%%%%%%%%%%%%%%%%%%%%%%%%%%%%%%%%%%%%%%%%%%%%%%%%%%%%%
\begin{document}

	%bibtex用の設定
	%\bibliographystyle{ujarticle}

	\twocolumn[
		\noindent
		\hspace{1em}
		2020 年 10 月 23 日
		ゼミ資料
		\hfill
		B4 杉山 竜弥
		\vspace{2mm}

		\hrule
		\begin{center}
			{\Large \bf 進捗報告}
		\end{center}
		\hrule
		\vspace{9mm}
	]

\section{今週やったこと}
\begin{itemize}
  \item 評価実験をたくさん
\end{itemize}

\section{実験}

10回行った探索に対し各々評価を1回, VGG19に対し異なるシード値で10回.
それぞれ10回ずつ実験した.

\begin{table}[tb]
  \begin{center}
    \caption{実験の設定}
    \begin{tabular}{|c|c|} \hline
      base model & VGG19 \\ \hline
      Optim($w$) & SGD(lr=0.0090131, momentum=0.9) \\ \hline
      % Optim($\alpha$) & Adam(lr=0.005, $\beta$=(0.5, 0.999)) \\ \hline
      Scheduler($w$) & Step($\gamma$=0.2344, stepsize=100) \\ \hline
      Loss & Cross Entropy Loss \\ \hline
      dataset & cifar10 \\ \hline
      batch size & 64 \\ \hline
      epoch & 150 \\ \hline
    \end{tabular}
    \label{tab:setting}
  \end{center}
\end{table}

表\ref{tab:setting}に評価時の実験設定を示した.
optunaによって得られた設定を利用した.

\subsection{結果}

評価時のグラフは./graphを参照.
表\ref{tab:acc}, \ref{tab:accg}にはテスト精度の結果を示した.

\begin{table*}[t]
  \begin{center}
    \caption{結果のテスト精度(\%)}
    \begin{tabular}{|c|c|c|c|c|c|c|c|c|c|c|}
    \hline
           & 1     & 2     & 3     & 4     & 5     & 6     & 7     & 8     & 9     & 10    \\ \hline
    評価     & 93.95 & 94.09 & 93.32 & 93.58 & 93.66 & 93.65 & 93.66 & 93.52 & 93.80 & 93.76 \\ \hline
    ベースライン & 92.97 & 92.95 & 93.25 & 92.90 & 93.06 & 92.93 & 93.07 & 93.07 & 93.03 & 93.06 \\ \hline
    \end{tabular}
    \label{tab:acc}
  \end{center}
\end{table*}

\begin{table*}[t]
  \begin{center}
    \caption{精度の比較}
    \begin{tabular}{|c|c|c|}
    \hline
           & \begin{tabular}[c]{@{}c@{}}test accracy\\ mean $\pm$ std \end{tabular} & delta    \\ \hline
    評価     & 93.6990 $\pm$ 0.2173                                                  & + 0.6700 \\ \hline
    ベースライン & 93.0290 $\pm$ 0.1002                                                  &          \\ \hline
    \end{tabular}
    \label{tab:accg}
  \end{center}
\end{table*}

\section{考察}
optunaでlr, $\gamma$, stepsizeを最適化したが, 期待していたlrが0.01, stepsizeが100という値に近いパラメータで$\gamma$が得られた. train sizeを20分の1にしていても割とうまくいくのかもしれない.

google colabだと3時間かかっていたのが, usagiサーバーだと1時間早くなった. GPUの性能かクラウドの同期に時間がかかるのかは不明だが, google colabで開発して, サーバーで実験を回すのが捗るかもしれない.

ベースラインに対して有意な差があることが分かったが, DARTによる探索の効果とショートカットの存在による効果が区別できないので, 何本か適当な位置にショートカット設けたランダムアーキテクチャとの比較も行いたい.


\section{今後の予定}
% なんとなくなんかの勉強をするとかではなく具体的に

\begin{itemize}
  % \item 重みと編集距離の変化の調査
  \item ランダムアーキテクチャとの比較
  \item DARTのunrolling実験
\end{itemize}

\section{ソースコード}
% 埋め込みでもGitでもいいので参照できるように
githubのnotebookリポジトリ参照.

% 参考文献リスト
\bibliographystyle{unsrt}
\bibliography{ref}
\end{document}
