%\documentstyle[epsf,twocolumn]{jarticle}       %LaTeX2e仕様
\documentclass[twocolumn]{jarticle}     %pLaTeX2e仕様(platex.exeの場合)
% \documentclass[onecolumn]{ujarticle}   %pLaTeX2e仕様(uplatex.exeの場合)
%%%%%%%%%%%%%%%%%%%%%%%%%%%%%%%%%%%%%%%%%%%%%%%%%%%%%%%%%%%%%%
%%
%%  基本バージョン
%%
%%%%%%%%%%%%%%%%%%%%%%%%%%%%%%%%%%%%%%%%%%%%%%%%%%%%%%%%%%%%%%%%
\setlength{\topmargin}{-45pt}
%\setlength{\oddsidemargin}{0cm}
\setlength{\oddsidemargin}{-7.5mm}
%\setlength{\evensidemargin}{0cm}
\setlength{\textheight}{24.1cm}
%setlength{\textheight}{25cm}
\setlength{\textwidth}{17.4cm}
%\setlength{\textwidth}{172mm}
\setlength{\columnsep}{11mm}

%\kanjiskip=.07zw plus.5pt minus.5pt


% 【節が変わるごとに (1.1)(1.2) … (2.1)(2.2) と数式番号をつけるとき】
%\makeatletter
%\renewcommand{\theequation}{%
%\thesection.\arabic{equation}} %\@addtoreset{equation}{section}
%\makeatother

%\renewcommand{\arraystretch}{0.95} 行間の設定
%%%%%%%%%%%%%%%%%%%%%%%%%%%%%%%%%%%%%%%%%%%%%%%%%%%%%%%%
%\usepackage{graphicx}   %pLaTeX2e仕様(\documentstyle ->\documentclass)
\usepackage[dvipdfmx]{graphicx}
\usepackage{subcaption}
\usepackage{multirow}
\usepackage{amsmath}
\usepackage{url}
\usepackage{ulem}
\usepackage{algorithm}
\usepackage{algorithmic}
\usepackage{listings} %,jlisting} %日本語のコメントアウトをする場合jlistingが必要
%ここからソースコードの表示に関する設定
\lstset{
  basicstyle={\ttfamily},
  identifierstyle={\small},
  commentstyle={\smallitshape},
  keywordstyle={\small\bfseries},
  ndkeywordstyle={\small},
  stringstyle={\small\ttfamily},
  frame={tb},
  breaklines=true,
  columns=[l]{fullflexible},
  numbers=left,
  xrightmargin=0zw,
  xleftmargin=3zw,
  numberstyle={\scriptsize},
  stepnumber=1,
  numbersep=1zw,
  lineskip=-0.5ex
}
\newcommand{\argmax}{\mathop{\rm arg~max}\limits}
\newcommand{\argmin}{\mathop{\rm arg~min}\limits}

%%%%%%%%%%%%%%%%%%%%%%%%%%%%%%%%%%%%%%%%%%%%%%%%%%%%%%%%
\begin{document}

	%bibtex用の設定
	%\bibliographystyle{ujarticle}

	\twocolumn[
		\noindent
		\hspace{1em}
		2020 年 12 月 18 日
		ゼミ資料
		\hfill
		B4 杉山 竜弥
		\vspace{2mm}

		\hrule
		\begin{center}
			{\Large \bf 進捗報告}
		\end{center}
		\hrule
		\vspace{9mm}
	]

\section{今週やったこと}
\begin{itemize}
  \item TDGAの理解
\end{itemize}

\section{TDGA}
$F$を最小化する.
\begin{equation}
  F = \langle E \rangle - HT
\end{equation}
\begin{equation}
  H = \sum^{M}_{k=1} H_k,
\end{equation}
\begin{equation}
  H_k = - \sum_{j} P^k_j log P^k_j
\end{equation}
エントロピー $H$ をグラフ用に改造する必要がある.

単純に考えるなら分散をとる?
\begin{equation}
  H = \frac{1}{|\mathcal{E}|} \sum_{i, j} ( \alpha_{ij} - \bar{\alpha}_{ij} )^2
\end{equation}
$|\mathcal{E}|$はDARTSの辺の数

\subsection{分からなかった点}
\begin{equation}
  \mathcal{P}(t+1, i, h) \Leftarrow \mathcal{P}(t+1) \cup \{h\}
\end{equation}
\begin{equation}
  F_h = \langle E_{\mathcal{P}(t+1, i, h)} \rangle - H_{\mathcal{P}(t+1, i, h)}T
\end{equation}
\begin{equation}
  \label{equ:mi}
  h_{\min} \Leftarrow \argmin_h F_h
\end{equation}
\begin{equation}
  \mathcal{P}(t+1) \Leftarrow \mathcal{P}(t+1) \cup \{h_{\min}\}
\end{equation}
(\ref{equ:mi}) 式は
どうやって $h$ を選んでいるのか.
$h$ は重複するのか

\section{今後の予定}
% なんとなくなんかの勉強をするとかではなく具体的に
TDGAを参考にしながら, GA(+DARTS)の実験

% 参考文献リスト
\bibliographystyle{unsrt}
\bibliography{ref}
\end{document}
