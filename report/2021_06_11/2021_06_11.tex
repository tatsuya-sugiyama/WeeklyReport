
%\documentstyle[epsf,twocolumn]{jarticle}       %LaTeX2e仕様
\documentclass[twocolumn]{jarticle}     %pLaTeX2e仕様(platex.exeの場合)
% \documentclass[onecolumn]{ujarticle}   %pLaTeX2e仕様(uplatex.exeの場合)
%%%%%%%%%%%%%%%%%%%%%%%%%%%%%%%%%%%%%%%%%%%%%%%%%%%%%%%%%%%%%%
%%
%%  基本バージョン
%%
%%%%%%%%%%%%%%%%%%%%%%%%%%%%%%%%%%%%%%%%%%%%%%%%%%%%%%%%%%%%%%%%
\setlength{\topmargin}{-45pt}
%\setlength{\oddsidemargin}{0cm}
\setlength{\oddsidemargin}{-7.5mm}
%\setlength{\evensidemargin}{0cm}
\setlength{\textheight}{24.1cm}
%setlength{\textheight}{25cm}
\setlength{\textwidth}{17.4cm}
%\setlength{\textwidth}{172mm}
\setlength{\columnsep}{11mm}

%\kanjiskip=.07zw plus.5pt minus.5pt


% 【節が変わるごとに (1.1)(1.2) … (2.1)(2.2) と数式番号をつけるとき】
%\makeatletter
%\renewcommand{\theequation}{%
%\thesection.\arabic{equation}} %\@addtoreset{equation}{section}
%\makeatother

%\renewcommand{\arraystretch}{0.95} 行間の設定
%%%%%%%%%%%%%%%%%%%%%%%%%%%%%%%%%%%%%%%%%%%%%%%%%%%%%%%%
%\usepackage{graphicx}   %pLaTeX2e仕様(\documentstyle ->\documentclass)
\usepackage[dvipdfmx]{graphicx}
\usepackage{subcaption}
\usepackage{multirow}
\usepackage{amsmath}
\usepackage{url}
\usepackage{ulem}
\usepackage{algorithm}
\usepackage{algorithmic}
\usepackage{listings} %,jlisting} %日本語のコメントアウトをする場合jlistingが必要
%ここからソースコードの表示に関する設定
\lstset{
  basicstyle={\ttfamily},
  identifierstyle={\small},
  commentstyle={\smallitshape},
  keywordstyle={\small\bfseries},
  ndkeywordstyle={\small},
  stringstyle={\small\ttfamily},
  frame={tb},
  breaklines=true,
  columns=[l]{fullflexible},
  numbers=left,
  xrightmargin=0zw,
  xleftmargin=3zw,
  numberstyle={\scriptsize},
  stepnumber=1,
  numbersep=1zw,
  lineskip=-0.5ex
}
\newcommand{\argmax}{\mathop{\rm arg~max}\limits}
\newcommand{\argmin}{\mathop{\rm arg~min}\limits}

%%%%%%%%%%%%%%%%%%%%%%%%%%%%%%%%%%%%%%%%%%%%%%%%%%%%%%%%
\begin{document}

	%bibtex用の設定
	%\bibliographystyle{ujarticle}

	\twocolumn[
		\noindent
		\hspace{1em}
		2021 年 6 月 11 日
		ゼミ資料
		\hfill
		M1 杉山 竜弥
		\vspace{2mm}

		\hrule
		\begin{center}
			{\Large \bf 進捗報告}
		\end{center}
		\hrule
		\vspace{9mm}
	]

  % \begin{table*}[tb]
  %   \begin{center}
  %     \caption{GPT-2の動作例}
  %     \begin{tabular}{|c|c|} \hline
  %       input & output \\ \hline
  %       2は偶数。3は奇数。5は & \verb|2は偶数。3は奇数。5は奇数(n≠0.0)である。3は...| \\ \hline
  %       2は偶数。3は奇数。8は & \verb|2は偶数。3は奇数。8は偶数は偶数?。11は偶数_。...|\\ \hline
  %     \end{tabular}
  %     \label{tab:exp}
  %   \end{center}
  % \end{table*}

\section{今週やったこと}
\begin{itemize}
  \item GPT-3の申請
  \item GPT-2の日本語学習済みモデル
\end{itemize}

\section{GPT-3の申請}
GPT-3はモデルをダウンロードするのではなく,
申請した後に, API を通して実行する形式のようであった.
実際に GPT-3 を動かしていこうとすると不便かもしれない.

もうすぐ申請してから 1 週間だが, 記入した内容がよくなかったのか
返信メールは今だ届かないため待機中.

\section{GPT-2の日本語学習済みモデル}
日本語学習済みモデルを探し, 先週のものより自然な文章生成ができることを確認できた.

\begin{itemize}
  \item input : 昔々あるところに、おじいさんとおばあさんが
  \item output : 住んでいました。
その部屋は小さく、ベッドがふたつあるだけでした。
ある時、おじいさんは部屋の扉をあけました。
すると、そこには何もなかったかのように、壁が現れました。
そしたら、何もなかったのです。
そんなある日、おじいさんはお母さんの家にお泊まりに行きました。
すると、お母さんは
このお泊り券はどうすれば、いいのだろう?
私はおじいさんとお話しすることにしました。
すると、なんとあのおじいさんは「よく、これを、受け取れ」と言うのです。
おじいさんは「どういう、ことなのだ?」とまだ半信半疑でいます。
「私たちが言っていたと思うが・・・、ここに住んでくれないか。」
おじいさんは
\end{itemize}

\url{https://github.com/tanreinama/gpt2-japanese}

\section{GPT-2のタスク}
表 \ref{tab:task} にGPT-2を利用することで解けるタスクの一部を示した.
文脈となる文章のほかに, このタスクの種類も入力する.
それぞれのタスク用に追加学習したモデルが多く公開されている(英語は).

TextGeneration はたくさん情報があるが,
それ以外のタスクで日本語を扱っているものが少ない.
おそらく見つけることができていないだけなので, エラーが出ず動くものを調査中.

\begin{table}[tb]
  \begin{center}
    \caption{GPT-2の解けるタスク}
    \begin{tabular}{|c|} \hline
      % AutomaticSpeechRecognition \\ \hline
      Conversational \\ \hline
      % FeatureExtraction \\ \hline
      % FillMask \\ \hline
      % ImageClassification \\ \hline
      QuestionAnswering \\ \hline
      Summarization \\ \hline
      TextClassification \\ \hline
      TextGeneration \\ \hline
      % TokenClassification \\ \hline
      Translation \\ \hline
      % ZeroShotClassification \\ \hline
      % Text2TextGeneration \\ \hline
      % TableQuestionAnswering \\ \hline
    \end{tabular}
    \label{tab:task}
  \end{center}
\end{table}
\url{https://huggingface.co/}

\section{Transformer encoder-decoderモデルベース雑談システムの
学習方法に対する主観評価の変動分析}
対話タスクについて : 確かに面白そうだと思いました.
一応英語ならすぐに動かせそうなので, このタスク周りについて考えてみます.

\section{今後の予定}
% なんとなくなんかの勉強をするとかではなく具体的に
\begin{itemize}
  \item いくつかのタスクでGPTを動かす
\end{itemize}

% 参考文献リスト
\bibliographystyle{unsrt}
\bibliography{ref}
\end{document}
